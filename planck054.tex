\documentclass[11pt,a4paper]{jsarticle}
	\usepackage{amsmath,amssymb}
	\usepackage{amsfonts}
	\usepackage{bm}
	\usepackage[dvipdfmx]{graphicx}
	\usepackage{ascmac}
	\usepackage{okumacro}
	\usepackage{titlesec}
	\usepackage{here}%画像を強制的に指定場所に置く[Here]を使える
	\usepackage{cases}%連立方程式など
	\setlength{\textwidth}{\fullwidth}
	\setlength{\textheight}{39\baselineskip}
	\addtolength{\textheight}{\topskip}
	\setlength{\voffset}{-0.5in}
	\setlength{\headsep}{0.3in}
	%
	\makeatletter
	\def\section{\@startsection {section}{1}{\z@}{-2.5ex plus -1ex minus -.2ex}{2.5 ex plus .2ex}{\LARGE\bf}}
	\def\subsection{\@startsection {subsection}{1}{\z@}{-1.5ex plus -1ex minus -.2ex}{2.3 ex plus .2ex}{\Large\bf}}
	\def\subsubsection{\@startsection {subsubsection}{1}{\z@}{-2.5ex plus -1ex minus -.2ex}{.3 ex plus .2ex}{\large \bf}}
	\makeatother

	\makeatletter
	    \renewcommand{\theequation}{%
	    \thesection.\arabic{equation}}
	    \@addtoreset{equation}{section}
	  \makeatother


	\newcommand{\divergence}{\mathrm{div}\,}  %ダイバージェンス
	\newcommand{\grad}{\mathrm{grad}\,}  %グラディエント
	\newcommand{\rot}{\mathrm{rot}\,}  %ローテーション
	\newcommand{\pt}{\partial}  %パーシャル
	\newcommand{\df}{\overset{\mathrm{def}}{=}} %def
	\newcommand{\non}{\nonumber}  %\nonumber
	\newcommand{\dis}{\displaystyle}
	\newcommand{\f}[1]{\framebox[1cm]{\textgt{\small #1}}}
	\newcommand{\kine}{\frac{1}{2}mv^2} %運動エネルギー
	\newcommand{\sceq}{Schrödinger equation}
	\pagestyle{plain}
		%\markright{\footnotesize \sf 物理数学1A 第一回中間テスト(2017) 解答例 \ %左上のタイトル
		%{@sakuPonit}} %名前
\begin{document}

%% 表紙
\begin{center}
    \huge 平成30年度 物理学実験(二)\par
    \vspace{15mm}
    \Huge プランク定数 \par
    \vspace{60mm}
    \Large  実験日: 平成30年5月28日\par
    \vspace{15mm}
    \Large 1217054 佐久間寛伸\par
    \vspace{10mm}
  \end{center}
  \thispagestyle{empty}
  \clearpage
  \addtocounter{page}{-1}
  \newpage

%% 目次
\tableofcontents
  \newpage

%% 内容
\section{目的}
	プランク定数測定器により,ハロゲン灯の光を分光し,振動数が知られているいくつかの単色光と
	Sb-Cs光電管とを用いて光電効果(photo-electric effect)の実験を行なう.それらの結果から,光が粒子性を示すことを理解し,
	さらにプランク定数(Planck constant)$h$を求める

\section{原理}
	\subsection{プランク定数 Planck constant}
		プランク定数とは「量子力学的な現象を特徴づける普遍定数。ドイツの物理学者プランクが熱放射の研究のなかで1900年に発見した。」\cite{planckconst01}
	\subsection{光電効果 photo-electric effect}
		金属の表面に光を照射すると電子が飛び出す.この現象を(外部)光電効果といい,
		飛び出した電子を光電子(photo-electron)と呼ぶ·アインシュタイン(A. Einstein)は, 1905年にプランク(M. Planck)
		のエネルギー量子の考え方(次節で説明)を発展させ,振動数$\nu$の光が伝播することは$hv$のエネルギーを持つ粒子が空間を
		飛んでいくことであると考えた.ここで$h$はプランク定数である.このような光の粒子を光子または光量子
		(photon) と名付けた.光電効果は,光子が金属中の電子と衝突してそれが持っているエネルギー を一度
		に全部電子に与え,その結果電子が金属の表面から外に飛び出す現象と解釈される.したがって,電子が金
		属表面を越えて外に出るために要する最小のエネルギーをeとすると,飛び出した光電子が持つことのでき
		る最大の運動エネルギーEは
		E = hv-eo,
		である.ここにeは電 の電荷の符号を変えたもので,素電荷と呼ばれる、またφは仕事関数
		function) とよばれ,金属の種類によって決まる
		(work
		1)でE=0,つまり
		を満たす振動数Voを限界振動数という.限界振動数以下の振動数v
		の金属表面から電子を飛び出させることはできない。
		Voの光は,それがどんなに強くても,そ
		光電管(photon-tube)に振動数がvの単色光を入射すると,受光面である陰極から電子が飛び出すので,こ
		の陰極(受光面) と陽極を結ぶ外の回路に電流が流れる.この電流を光電流(photo-current) と呼ぶ.このと
		き,陽極側が負になるように電圧をかけ,その電圧を増加して行くと,光電流は減少する.この負電圧と光
		電流の関係を表す曲線を外挿して光電流がゼロに対応する負電圧をもとめる.このとき,光電子がどんな運
		動エネルギーを持っていても,全て陰極に押し返されている.したがって,この負電圧を-hとすると,E=
		ehであるから,式(1)は,
		となる·いくつかの振動数νについて同様な測定を行い,vとећの関係を求めると,それは直線となり,その
		勾配からプランク定数hが得られる·知られている の值は(6.626176 ±0.000036)×10-34Jsである.また,
		の直線を外挿して,用いた光電管の受光面を形成している金属の表面物質の仕事関数φが得られる.
		光がhvのエネルギーを持つ粒子であることから,単色光を指定する場合に波長や振動数のほか,光子エネ
		ルギー(photon energy)も使用される.単位としては,通常,エレクトロン·ボルト (eV)を用いる.1エレク
		トロン·ボルト=ex1ジュールである
	\subsection{エネルギー量子 energy quantum}
		M.プランクは空洞放射の測定値を十分説明できる関係式 (プランクの放射式 ) を見出したが,この式を導き出すには,振動数 ν の振動子のエネルギーの放出・吸収が連続的ではなく,hν を単位とする不連続な量の放出・吸収だけが許される,と仮定せざるをえなかった。ただし,h はプランク定数である。この意味で,hν をエネルギー量子という。エネルギー量子の考えは,エネルギーの連続性を根本的な足場にしている古典物理学の自然観と正面から対立し,量子論を生み出す第一歩となった。この功績により,1918年プランクにノーベル物理学賞が授与された。プランクのエネルギー量子という考えは,A.アインシュタインによって光量子という考えに発展した。





%% 参考文献
\begin{thebibliography}{数字}
  \bibitem{planckconst01} 小学館 日本大百科全書(ニッポニカ) 江沢 洋
  \bibitem{キー2} 参考文献の名前・著者2
  \bibitem{キーN} 参考文献の名前・著者N
	\end{thebibliography}
\end{document}
